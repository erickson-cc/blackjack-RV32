\documentclass[ ]{article}
\usepackage[ ]{indentfirst}
\usepackage[ ]{helvet}
\usepackage[ ]{xcolor}
\usepackage[ ]{listings}

\lstset{
	tabsize = 4,
	breaklines=true,
	backgroundcolor= \color{lightgray},
}
\begin{document}
	\section{Lógica de Cartas}
	O baralho do jogo foi declarado na memória usando a diretiva \textit{.byte}. Isso facilita no manuseio do baralho pois cada carta possui 3 informações intrínsecas: seu número, seu valor e sua quantidade. 
	
	Dessa forma, o valor da carta estará no espaço de memória dedicado ao rótulo \textit{baralho\_valores}, a quantidade de cartas estará no espaço de memória dedicado ao rótulo \textit{baralho\_qtd } e o número da carta será dado de acordo com o índice em ambos os espaços.
	
	
	\begin{lstlisting}
.data	0x10010000
baralho_qtd:
	.byte	4, 4, 4, 4, 4, 4, 4, 4, 4, 4, 4, 4, 4
baralho_valores:
	.byte	1, 2, 3, 4, 5, 6, 7, 8, 9, 10, 10, 10, 10
	\end{lstlisting}
	
	Para acessar uma determinada carta sorteia-se um índice de 0 a 12, e, a partir do espaço de memória \textit{baralho\_qtd} endereçado no registrador $t0$, verifica se a quantidade é maior que 0 e decrementa esse valor. 
	\begin{lstlisting}
sortear:
	li a0, 0
	li a1, 13
	li a7, 42
	ecall

	add t1, t0, a0
	lb t2, 0(t1) 
	beqz t2, sortear 
	

	addi t2, t2, -1
	sb t2, 0(t1) 
	ret
	\end{lstlisting}
	\textbf{Falta implementar uma lógica parecida com o \textit{baralho\_valores}}
\end{document}